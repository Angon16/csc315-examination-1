
\chapter{Aeint Thet Ngon}

\begin{enumerate}
  \item Most programming languages require the use of brackets to
    enclose the index in a reference to an element of an array.
  \begin{enumerate}
    \item Identify a language the requires the use of parentheses
      to enclose the index in a reference to an element of an array.
    \item Why did the designers of the language choose parentheses
      rather than brackets?
    \end{enumerate}

  \begin{answer}

  \begin{enumerate}
    \item Ada
    \item Because the designers wanted uniformity between array references and function calls in expressions in spite of potential readability problems. 
    \end{enumerate}

    \end{answer}
    
  \item What is the relationship between a lexeme and a token?

  \begin{answer}

    A lexeme is a formal descriptions of a syntax of progamming languages and most of the times do not include descriptions of the lowest-level syntactic units. Each lexeme group is represented by a name or a token. So a token of a language is a category of its lexemes.

    \end{answer}

  \item
  \begin{enumerate}
    \item What kind of symbols are found at the internal nodes of a
      parse tree?
    \item What kind of symbols are found at the leaves of a parse tree?
    \end{enumerate}

  \begin{answer}

  \begin{enumerate}
    \item Non-terminal cateogires of the grammar are found at the internal nodes of a parse tree.
    \item Leaf nodes are labelled by terminal categories.
    \end{enumerate}

    \end{answer}


  \item One of the most significant contributions from the developers
    of ALGOL 60 also limited the success of that language. What was
    that contribution?

  \begin{answer}

    BNF, one of the most important contributions to computer science, is considered a simple and elegant means of syntax description but in 1960 it seemed strange and complicated and was a factor in its lack of acceptance.

    \end{answer}

  \item What problem were the creators of Common LISP trying to solve?

  \begin{answer}

    During the 1970s and early 198s, due to the usage of diverse dialects of LISP, there was a problem of lack of portability among the programs writting using different dialects. To solve the problem, Common Lisp was created by combining the features of different dialects of LISP.

    \end{answer}

  \item What is an ambiguous context free grammar?

  \begin{answer}

    A grammar that genereates a sentential form for which there are two or more distinct parse trees is said to be ambiguous.

    \end{answer}

  \item Contrast the complexity of algorithms that can parse strings
    that conform to the most general kinds of context free grammars
    and the complexity of the algorithms that can parse strings that
    conform to the grammars of programming languages?

  \begin{answer}

    Parsing algorithms that work for any unambiguous grammar are complicated and inefficient, the amount of time they take is on the order of the cube of the lenght of the string to be parsed. So generality is traded for efficiency. Faster alogrithms have been found that work only for a subset of the set of all possible grammars and the time they take is linearly related to the length of the string to be parsed.

    \end{answer}

  \item Java represents characters with Unicode. It is the first
    widely used programming language with this feature. What is the
    significance of this feature?

  \begin{answer}

    This feature includes the characters from most of the world's natural languages. It was developed because of globalization of business and the need for computers to communicate with other computers around the world.

    \end{answer}

  \item How does the binary coded decimal type differ from the
    floating point type?

  \begin{answer}

    Floating point data types model real numbers but the representation are only apprximations for many real values. Decimal types have the advantage of being able to precisely store decimal values, which cannot be done with floating point.

    \end{answer}

  \item Identify a user-defined ordinal type in the Java programming
    language.

  \begin{answer}

    Enumeration Types

    \end{answer}

  \item Mathematicians and programmers might have different ideas
    about the precedence of Boolean operators. Explain.

  \begin{answer}

    In mathematics, Boolean algebras have equal precedence, however, C-based languages assign higher precedence to AND than OR. This might have resulted from the baseless correlation of mulitplcation with AND and of additon with OR, which would then naturally assign higher precedence to AND.

    \end{answer}

  \item Programmers should use \verb+===+ rather than \verb+==+ to
    test the equality of the values of two expressions in JavaScript. Why?

  \begin{answer}

    Because whena  string and a number are the operands of a relational operator, the string is coerced to a number but when \verb+===+ is used no coercion is done on the operands of this operator.

    \end{answer}

  \item Describe a hazard of allowing short-circuited evaluation
    of expressions and side effects in expressions at the same time.

  \begin{answer}

    A language that provides short-circuit evaluations of Boolean expressions and also has side effects in expressions allows subltel errors to occr. Suppose that short-circuit evaluation is used on an expression and part of the expression that contains a side effect is not evaluated; then the side effect will occur only in complete evaluations of the whole expression. If a program correctness depends on the side effect, short-circuit evaluation can result in a serious error.

    \end{answer}

  \item Briefly describe the three steps in the mark-sweep algorithm
    for garbage collection.

  \begin{answer}

    \begin{itemize}
      \item{All cells in the heap have their indicators set to indicate they are garbage}
      \item{Marking phase-Every pointer in the program is traced into the heap and all reachable cells are marked as not being garbage}
      \item{Sweep phase-all cells in the heap that have not been specifially marked as still being used are returned to the list of available space}
    \end{itemize}

    \end{answer}

  \item What led Yukihiro Matsumoto to create the Ruby programming language?

  \begin{answer}

    Yukihiro Matsumoto's motivation was due to his dissatisfaction with Perl and Python, which support object-oriented programming but neither is puer object oriented langauge.

    \end{answer}

  \item What did Microsoft aim to achieve with its development of the
    C\# language?

  \begin{answer}

    The purose of C\# is to provide a language for componenet based software development. In this environment, components from a variety of languages can be easily combined to form systems. 

    \end{answer}

  \end{enumerate}



\section{More questions for discussion and review.}

\begin{enumerate}
  \item The design of which machine influenced the design
    of the control statements in FORTRAN?

  \item How many different kinds of control statements
    must the designer of a programming language include
    in a language?

  \item What is the one question that applies in the
    design of all statements that allow selection or
    iteration?

  \item What is an advantage of requiring that
    the \textbf{then} and \textbf{else} clauses of
    an \textbf{if} statement be compound statements?

  \item How does the \textbf{switch} statement in C\#
    differ from the \textbf{switch} statement in Java?

  \item Distinguish between 2 statements in Ruby
    that correspond to Java's \textbf{switch} statement.

  \item Features of a programming language sometimes persist
    longer than a feature of computing hardware that inspired
    and supported that part of the language's design.
    Similarly, features of hardware sometimes persist longer
    than some parts of a language's design that were created
    to take advantage of that feature in hardware.

    Give examples.

  \item Who most famously warned of the dangers of using the
    \textbf{goto} statement? What did Donald Knuth have to
    say about the use of the \textbf{goto} statement?

  \item Describes Ada's \textbf{for} loop. Are there some
    kinds of iteration that might be easier in Ada than
    in Java? Easier in Java than in Ada?

  \item What does it mean to say that the guarded commands
    of Ada are non-deterministic?

  \item The header files in a C program contain function
    prototypes. What is a function prototype?

  \item Every method in a Ruby program belongs to a class.
    A programmer can place a definition of a method inside
    the definition of a class or outside of the definition
    of any class that the programmer writes. To which class
    does the method belong in the second case?

  \item Distinguish between positional and keyword parameters.

  \item Ruby blocks are closures. What does that mean?

  \item What is a pure function?

  \item Some languages give programmers means to define
    both functions and procedures. Java doe not. Is that
    a serious limitation?

  \item Declarations of formal parameters in an Ada procedure
    can include, in addition to the names and types of the
    parameters, reserved words that do not appear in declarations
    in Java programs. 
    What is the purpose of those reserved words?
 
  \item The C language imposes a constraint upon programmers
    who want to pass a multidimensional array to a function.
    What is the constraint? How did the design of the Java
    programming language eliminate that constraint for 
    programmers who use that language?

  \item An activation record contains a return
    address, a dynamic link, parameters, and
    local variables.
  \begin{enumerate}
    \item To what does the return address point?
    \item To what does the dynamic link point?
    \end{enumerate}

  \item The stack will contain multiple activation
    records for a single subprogram under what
    circumstances?

  \item How (or why?) does the LIFO protocol apply to
    calls to and returns from subprograms?

  \item Which important development in computer architecture
    has changed the way that the stack is used in some
    systems for facilitating calls to and returns from
    subprograms?

  \item A dynamic chain contains a history of what?

  \item Which two numbers are needed to compute
    the address of a local variable in a subprogram?

  \item How does a Ruby module differ from a class?

  \item Memory for variables can be allocated on the heap
    and on the stack. In which place or places is memory
    allocated for objects in C++? in Java?

  \item What problems were solved by the addition
    of genericity to Java?

  \item What is the purpose of the static chain?

  \item What is a singleton?

  \item What are the two parts of the definition 
    of an abstract data type?

  \end{enumerate}


