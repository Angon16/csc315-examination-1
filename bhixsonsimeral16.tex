
\chapter{Brian Hixson-Simeral}

\begin{enumerate}
  \item Most programming languages require the use of brackets to
    enclose the index in a reference to an element of an array.
  \begin{enumerate}
    \item Identify a language the requires the use of parentheses
      to enclose the index in a reference to an element of an array.
    \item Why did the designers of the language choose parentheses
      rather than brackets?
    \end{enumerate}

  \begin{answer}

  \begin{enumerate}
    \item Ada uses parentheses to enclose the index to an element of an array.
    \item The reason that brackets are used, rather than parentheses,
      is that parentheses are also used to denote subprogram calls.
    \end{enumerate}

    \end{answer}
    
  \item What is the relationship between a lexeme and a token?

  \begin{answer}

    Tokens are the catagorey of the lexeme.  Eg. 2 and int\_literal
    (lexeme and token).

    \end{answer}

  \item
  \begin{enumerate}
    \item What kind of symbols are found at the internal nodes of a
      parse tree?
    \item What kind of symbols are found at the leaves of a parse tree?
    \end{enumerate}

  \begin{answer}

  \begin{enumerate}
    \item Nonterminal
    \item Terminal
    \end{enumerate}

    \end{answer}


  \item One of the most significant contributions from the developers
    of ALGOL 60 also limited the success of that language. What was
    that contribution?

  \begin{answer}

    BNF (Backus-Naur Form)

    \end{answer}

  \item What problem were the creators of Common LISP trying to solve?

  \begin{answer}

    They were trying to create one version of LISP, so that there
    wouldn't be so many dialects being used.  With the large amount of
    dialects came a lack of potability.

    \end{answer}

  \item What is an ambiguous context free grammar?

  \begin{answer}

    A context-free grammar is a generative device for defining
    languages. Context-free gramars are ambiguous when they generate a
    sentential form that could have two or more distinct parse trees.

    \end{answer}

  \item Contrast the complexity of algorithms that can parse strings
    that conform to the most general kinds of context free grammars
    and the complexity of the algorithms that can parse strings that
    conform to the grammars of programming languages?

  \begin{answer}

    A parsing algorithm for an unambiguous grammar is ridiculously
    inefficient ($O(n^3)$).  More specifc alogorithms can be made for
    programming languages that have a complexity of $O(n)$.  It is much
    more efficient to use specific algorithms.

    \end{answer}

  \item Java represents characters with Unicode. It is the first
    widely used programming language with this feature. What is the
    significance of this feature?

  \begin{answer}

    It was a 16-bit character set that included characters from most
    natural languages and ASCII.

    \end{answer}

  \item How does the binary coded decimal type differ from the
    floating point type?

  \begin{answer}

    Floating-point are represented as fractions and exponents while
    Decimal are stored with a fixed number of decimal digits with the
    decimal point at a fixed position in the value.  The value 0.1 can
    be represented exactly in decimal, but in floating-point it would
    come with some uncertainty.

    \end{answer}

  \item Identify a user-defined ordinal type in the Java programming
    language.

  \begin{answer}

    The two user-defined ordinal types in Java are enumeration and subrange.

    \end{answer}

  \item Mathematicians and programmers might have different ideas
    about the precedence of Boolean operators. Explain.

  \begin{answer}

    In math, AND and OR have equal precedence, but in most programming
    languages AND has a higher precedence than OR.

    \end{answer}

  \item Programmers should use \verb+===+ rather than \verb+==+ to
    test the equality of the values of two expressions in JavaScript. Why?

  \begin{answer}

    Programmers should use === instead of == because when you use == a
    string such as "7" will be coerced to the number 7, but when you
    use === it remains a string.

    \end{answer}

  \item Describe a hazard of allowing short-circuited evaluation
    of expressions and side effects in expressions at the same time.

  \begin{answer}

    This allows subtle errors to occur.  "If the short-circuit
    evaluation is used on an expression and part od the expression
    that contains a side effect is not evaluated; the the side effect
    will occur only in complete evaluations of the wole expression.
    If program correctness depends on the side effect, short-circuit
    evaluation can result in a serious error."

    \end{answer}

  \item Briefly describe the three steps in the mark-sweep algorithm
    for garbage collection.

  \begin{answer}

    All cells in the heap have their indicators set to indicate that
    they are garbage.  Every pointer in the program is traced into the
    heap, and all reachable cells are marked as not being garbage.
    All cells in the heap that are marked as garbage are returned to
    the list of available space.

    \end{answer}

  \item What led Yukihiro Matsumoto to create the Ruby programming language?

  \begin{answer}

    He was dissatisfied with Perl and Python.  He wanted a purely
    object-oriented language and neither of them lived up to it.

    \end{answer}

  \item What did Microsoft aim to achieve with its development of the
    C\# language?

  \begin{answer}

    C\# was meant to provide a language for component-based software
    development.  It was geared towards development in the .NET
    Framework.

    \end{answer}

  \end{enumerate}


