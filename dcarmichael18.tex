
\chapter{Dot Carmichael}

\begin{enumerate}
  \item Most programming languages require the use of brackets to
    enclose the index in a reference to an element of an array.
  \begin{enumerate}
    \item Identify a language the requires the use of parentheses
      to enclose the index in a reference to an element of an array.
    \item Why did the designers of the language choose parentheses
      rather than brackets?
    \end{enumerate}

  \begin{answer}

  \begin{enumerate}
    \item Ada uses parentheses to enclose the index.
    \item The programmers used parentheses instead of brackets to provide
          uniformity between arrays and functions, since they both use
          mapping.
    \end{enumerate}

    \end{answer}
    
  \item What is the relationship between a lexeme and a token?

  \begin{answer}

    A token is a category of one or more lexemes within a language.

    \end{answer}

  \item
  \begin{enumerate}
    \item What kind of symbols are found at the internal nodes of a
      parse tree?
    \item What kind of symbols are found at the leaves of a parse tree?
    \end{enumerate}

  \begin{answer}

  \begin{enumerate}
    \item Non-terminal symbols.
    \item Terminal symbols.
    \end{enumerate}

    \end{answer}


  \item One of the most significant contributions from the developers
    of ALGOL 60 also limited the success of that language. What was
    that contribution?

  \begin{answer}

    Backus-Naur Form syntax description

    \end{answer}

  \item What problem were the creators of Common LISP trying to solve?

  \begin{answer}

    The problem was a lack of portability between programs written in
    different dialects of LISP.

    \end{answer}

  \item What is an ambiguous context free grammar?

  \begin{answer}

    An ambiguous context free grammar can generate sentences with more than
    one leftmost and rightmost derivation.

    \end{answer}

  \item Contrast the complexity of algorithms that can parse strings
    that conform to the most general kinds of context free grammars
    and the complexity of the algorithms that can parse strings that
    conform to the grammars of programming languages?

  \begin{answer}

    Algorithms for parsing any unambiguous grammars are highly inefficient,
    with a complexity of O(n**3), whereas algorithms for programming languages
    are complexity O(n).

    \end{answer}

  \item Java represents characters with Unicode. It is the first
    widely used programming language with this feature. What is the
    significance of this feature?

  \begin{answer}

    Unicode allows programs to be written which can communicate with other
    countries whose languages cannot be represented by the basic english
    alphabet.

    \end{answer}

  \item How does the binary coded decimal type differ from the
    floating point type?

  \begin{answer}

    Floating point numbers are an approximation of the number using fractions
    and exponents, stored in binary which can make them even less accurate.
    Binary coded decimals precisely store decimal values at the cost of taking
    up more storage than floating point numbers.

    \end{answer}

  \item Identify a user-defined ordinal type in the Java programming
    language.

  \begin{answer}

    Enumeration

    \end{answer}

  \item Mathematicians and programmers might have different ideas
    about the precedence of Boolean operators. Explain.

  \begin{answer}

    When Boolean operators evaluate to 0 and 1, they can be used in equations
    with multiple relational operators to produce unexpected results.
    (1 < 2 <= 1) -> ((true/1) <= 1) -> true

    \end{answer}

  \item Programmers should use \verb+===+ rather than \verb+==+ to
    test the equality of the values of two expressions in JavaScript. Why?

  \begin{answer}

    With +==+, strings can be coerced into numbers to check for equality.
    With +===+, no coercion will take place, so strings and numbers will not
    resolve as equal.

    \end{answer}

  \item Describe a hazard of allowing short-circuited evaluation
    of expressions and side effects in expressions at the same time.

  \begin{answer}

    If there is a side effect in the right hand side of the expression and
    it short-circuits on the left, then the side effect will not occur.

    \end{answer}

  \item Briefly describe the three steps in the mark-sweep algorithm
    for garbage collection.

  \begin{answer}

    1- all cells in heap set indicators to 'garbage'
    2- all pointers in the program are traced, and anything which is pointed
       to is changed to 'not garbage'
    3- all cells still marked 'garbage' are reverted to available space

    \end{answer}

  \item What led Yukihiro Matsumoto to create the Ruby programming language?

  \begin{answer}

    Matsumoto was dissatisfied with Perl and Python; specifically, that they
    were not pure object-oriented languages because they had primitive types
    and functions.

    \end{answer}

  \item What did Microsoft aim to achieve with its development of the
    C\# language?

  \begin{answer}

    Microsoft aimed to provide a language for development of component-based
    software in the .NET framework.

    \end{answer}

  \end{enumerate}


