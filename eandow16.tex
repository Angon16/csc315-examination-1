
\chapter{Eric Andow}

\begin{enumerate}
  \item 1. Most programming languages require the use of brackets to
    enclose the index in a reference to an element of an array.
  \begin{enumerate}
    \item Identify a language the requires the use of parentheses
      to enclose the index in a reference to an element of an array.
    \item Why did the designers of the language choose parentheses
      rather than brackets?
    \end{enumerate}

  \begin{answer}

  \begin{enumerate}
    \item Ada is one example that uses parentheses to reference array elements.
    \item In the case of Ada, the designers chose to use a uniform
      style for function calls and array references because,
      ultimately, both represent a mapping to a specific point in
      memory.
    \end{enumerate}

    \end{answer}
    
  \item 2. What is the relationship between a lexeme and a token?

  \begin{answer}

    A lexeme is an association that the compiler uses to understand
    your code. So, a number is understood as an integer literal and an
    equal sign is understood as an equivalence operator. The lexemes
    are then grouped into categories, called tokens. While a lexeme
    appears verbatim in your code, a token is a name that only the
    compiler uses, like 'equal\_sign' or 'mult\_op'.

    \end{answer}

  \item 3.
  \begin{enumerate}
    \item What kind of symbols are found at the internal nodes of a
      parse tree?
    \item What kind of symbols are found at the leaves of a parse tree?
    \end{enumerate}

  \begin{answer}

  \begin{enumerate}
    \item The internal nodes of a parse tree are non-terminal symbols,
      which branch out into other symbols.
    \item At the leaves of the tree lie terminal symbols, each of
      which represents an atom of the original code.
    \end{enumerate}

    \end{answer}


  \item 4. One of the most significant contributions from the developers
    of ALGOL 60 also limited the success of that language. What was
    that contribution?

  \begin{answer}

    ALGOL 60 heralded the first use of the BNF formalism. BNF stands
    for Backus-Naur form, one of the primary ways of describing the
    syntax of programming languages. Unfortunately, when it was
    introduced, BNF was seen as too complicated. The slow shift
    towards accepting BNF would come too late for ALGOL 60 to ever
    gain a foothold among users.

    \end{answer}

  \item 5. What problem were the creators of Common LISP trying to solve?

  \begin{answer}

    As the Reagan era commenced, Americans were facing tougher
    problems than they knew how to deal with. Namely, they had too
    many dialects of Lisp. Like having too many dialects of any
    language, this caused widespread strife and factionalism among
    those who were left with no way to communicate. To bring their
    colleagues back to the table of brotherhood, a few brave
    developers created Common Lisp, a language which included features
    of several popular Lisp dialects. In this way, the developers gave
    the many dialects a unified way to communicate.

    \end{answer}

  \item 6. What is an ambiguous context free grammar?

  \begin{answer}

    Grammar?? But seriously, an ambiguous grammar is a sort of context-free grammar (one for which a parse tree can be generated) which can produce more than one valid parse tree for some input.

    \end{answer}

  \item 7. Contrast the complexity of algorithms that can parse strings
    that conform to the most general kinds of context free grammars
    and the complexity of the algorithms that can parse strings that
    conform to the grammars of programming languages?

  \begin{answer}

    The most general kinds of context-free grammar require parsers
    that are of complexity O(n cubed), because they often make
    mistakes and must rebuild parts of the parse tree. Any
    commercially viable algorithm (the kind used to parse programming
    languages for compilers) will only operate on a subset of the
    grammars, but it will be of complexity O(n).

    \end{answer}

  \item 8. Java represents characters with Unicode. It is the first
    widely used programming language with this feature. What is the
    significance of this feature?

  \begin{answer}

    Most languages support ASCII, a character set which only includes
    the english alphabet, plus special characters. UCS-2 is a standard
    developed by the Unicode Consortium which supports most character
    and number systems from around the world. This means that
    programmers in Thailand don't have to write custom Java to
    interpret their input; it just works.

    \end{answer}

  \item 9. How does the binary coded decimal type differ from the
    floating point type?

  \begin{answer}

    A binary coded decimal represents a set of decimal digits to store
    a decimal number precisely. In contrast, a floating-point number
    stores a less-precise value and an exponent. This means that
    floating-point numbers have a much wider range, and take up less
    memory, but they are less precise (1.0 - 1) might give you
    something like (0.000000001).

    \end{answer}

  \item 10. Identify a user-defined ordinal type in the Java programming
    language.

  \begin{answer}

    In 2004, Java added enumeration types, which map a set of discrete
    values onto a subset of the integers. The user can match any set
    of labels invisibly to a set of distinct integers.

    \end{answer}

  \item 11. Mathematicians and programmers might have different ideas
    about the precedence of Boolean operators. Explain.

  \begin{answer}

    For some reason, C-based languages give the AND operator
    precedence over the OR operator. However, mathematicians will
    remind you over and over again that the two operators should have
    equal precedence. They invented Boolean algebra, for crying out
    loud!

    \end{answer}

  \item 12. Programmers should use \verb+===+ rather than \verb+==+ to
    test the equality of the values of two expressions in JavaScript. Why?

  \begin{answer}

    In JavaScript, == means test whether the two sides evaluate to the
    same thing. So, (5+1) == (9-3) returns true. However, this isn't
    very useful if we want to match the exact expression, for example
    we want to find a difference between '7' and 7. For those
    purposes, we use ===, which does not affect the operands before
    comparing them.

    \end{answer}

  \item 13. Describe a hazard of allowing short-circuited evaluation
    of expressions and side effects in expressions at the same time.

  \begin{answer}

    Short-circuited evaluations can evaluate only the left-hand side,
    so if there are side-effects on the right-hand side (as in 'latest
    > bestSoFar || index++ > size), they might not happen in some
    cases. When the program decides on side effects at runtime, this
    can be really difficult to track.

    \end{answer}

  \item 14. Briefly describe the three steps in the mark-sweep algorithm
    for garbage collection.

  \begin{answer}

    Only run this when you are out of available memory.
    1. Mark everything to look like garbage.
    2. For each pointer, mark what it points to as not-garbage.
    3. All cells that are still marked as garbage are reclaimed. 

    \end{answer}

  \item 15. What led Yukihiro Matsumoto to create the Ruby programming language?

  \begin{answer}

    Matz was unhappy with the neat languages that everyone liked, like
    Perl and Python. While they allowed you to use objects if you
    wanted, Matz thought that this was too weak. He wanted a language
    that would scale without the inelegance of primitive types or
    functions. Therefore, he built Ruby in the model of Smalltalk: a
    purely object-oriented language, where even primitive operators
    were methods and could be re-written.

    \end{answer}

  \item 16. What did Microsoft aim to achieve with its development of the
    \verb+C#+ language?

  \begin{answer}

    Microsoft wanted all the cool cats to use their .Net framework, so
    that they could obtain the power of huge market share. To achieve
    their dreams, they forged \verb+C#+ a weapon meant to combine the
    power of Java and C++ into a new general purpose programming
    language. They intended this great power to attract the greedy
    race of men, who would swarm in countless legions under the banner
    of .Net.

    \end{answer}

  \end{enumerate}



\section{More questions for discussion and review.}

\begin{enumerate}
  \item The design of which machine influenced the design
    of the control statements in FORTRAN?

  \item How many different kinds of control statements
    must the designer of a programming language include
    in a language?

  \item What is the one question that applies in the
    design of all statements that allow selection or
    iteration?

  \item What is an advantage of requiring that
    the \textbf{then} and \textbf{else} clauses of
    an \textbf{if} statement be compound statements?

  \item How does the \textbf{switch} statement in C\#
    differ from the \textbf{switch} statement in Java?

  \item Distinguish between 2 statements in Ruby
    that correspond to Java's \textbf{switch} statement.

  \item Features of a programming language sometimes persist
    longer than a feature of computing hardware that inspired
    and supported that part of the language's design.
    Similarly, features of hardware sometimes persist longer
    than some parts of a language's design that were created
    to take advantage of that feature in hardware.

    Give examples.

  \item Who most famously warned of the dangers of using the
    \textbf{goto} statement? What did Donald Knuth have to
    say about the use of the \textbf{goto} statement?

  \item Describes Ada's \textbf{for} loop. Are there some
    kinds of iteration that might be easier in Ada than
    in Java? Easier in Java than in Ada?

  \item What does it mean to say that the guarded commands
    of Ada are non-deterministic?

  \item The header files in a C program contain function
    prototypes. What is a function prototype?

  \item Every method in a Ruby program belongs to a class.
    A programmer can place a definition of a method inside
    the definition of a class or outside of the definition
    of any class that the programmer writes. To which class
    does the method belong in the second case?

  \item Distinguish between positional and keyword parameters.

  \item Ruby blocks are closures. What does that mean?

  \item What is a pure function?

  \item Some languages give programmers means to define
    both functions and procedures. Java doe not. Is that
    a serious limitation?

  \item Declarations of formal parameters in an Ada procedure
    can include, in addition to the names and types of the
    parameters, reserved words that do not appear in declarations
    in Java programs. 
    What is the purpose of those reserved words?
 
  \item The C language imposes a constraint upon programmers
    who want to pass a multidimensional array to a function.
    What is the constraint? How did the design of the Java
    programming language eliminate that constraint for 
    programmers who use that language?

  \item An activation record contains a return
    address, a dynamic link, parameters, and
    local variables.
  \begin{enumerate}
    \item To what does the return address point?
    \item To what does the dynamic link point?
    \end{enumerate}

  \item The stack will contain multiple activation
    records for a single subprogram under what
    circumstances?

  \item How (or why?) does the LIFO protocol apply to
    calls to and returns from subprograms?

  \item Which important development in computer architecture
    has changed the way that the stack is used in some
    systems for facilitating calls to and returns from
    subprograms?

  \item A dynamic chain contains a history of what?

  \item Which two numbers are needed to compute
    the address of a local variable in a subprogram?

  \item How does a Ruby module differ from a class?

  \item Memory for variables can be allocated on the heap
    and on the stack. In which place or places is memory
    allocated for objects in C++? in Java?

  \item What problems were solved by the addition
    of genericity to Java?

  \item What is the purpose of the static chain?

  \item What is a singleton?

  \item What are the two parts of the definition 
    of an abstract data type?

  \end{enumerate}


