
\chapter{Nicci Geiger}

\begin{enumerate}
  \item Most programming languages require the use of brackets to
    enclose the index in a reference to an element of an array.
  \begin{enumerate}
    \item Identify a language the requires the use of parentheses
      to enclose the index in a reference to an element of an array.
    \item Why did the designers of the language choose parentheses
      rather than brackets?
    \end{enumerate}

  \begin{answer}

  \begin{enumerate}
    \item Ada uses parentheses to enclose the index refrence to the element of an array.
    \item This was done for uniformity between array refrences and function call expressions because they are both mappings.
    \end{enumerate}

    \end{answer}
    
  \item What is the relationship between a lexeme and a token?

  \begin{answer}

    A lexemes is the lowest-level syntatic units.  Lexemes are patitioned into groups which are represented by tokens.
    \end{answer}

  \item
  \begin{enumerate}
    \item What kind of symbols are found at the internal nodes of a
      parse tree?
    \item What kind of symbols are found at the leaves of a parse tree?
    \end{enumerate}

  \begin{answer}

  \begin{enumerate}
    \item Nonterminal symobols are found at internal nodes.
    \item Terminal symbols  are found at leaves.
    \end{enumerate}

    \end{answer}


  \item One of the most significant contributions from the developers
    of ALGOL 60 also limited the success of that language. What was
    that contribution?

  \begin{answer}

  The contribution was BNF (Backus-Naur Form) a natural notation for describing syntax.

    \end{answer}

  \item What problem were the creators of Common LISP trying to solve?

  \begin{answer}

    They were trying solve the issue of lack of portability in programs written in various dialects.

    \end{answer}

  \item What is an ambiguous context free grammar?

  \begin{answer}

   A grammar that generates a sentential for that has two or more parse trees.

    \end{answer}

  \item Contrast the complexity of algorithms that can parse strings
    that conform to the most general kinds of context free grammars
    and the complexity of the algorithms that can parse strings that
    conform to the grammars of programming languages?

  \begin{answer}

    The complexity of context free grammars is O(n^3), while the grammars of programing languages have a complexity of O(n) due to the fact that context free grammar parsers must frequently be backed up and reparse part of what is being analyzed.

    \end{answer}

  \item Java represents characters with Unicode. It is the first
    widely used programming language with this feature. What is the
    significance of this feature?

  \begin{answer}

   It uses a set of charactes that covers characters for most of the worlds alphabet.  natural languages. This caters to the need for global computer communication.

    \end{answer}

  \item How does the binary coded decimal type differ from the
    floating point type?

  \begin{answer}

    Float points type represent real numbers but are only approximate and are stored in binary and represented in fractions or exponents, while decimal type stores a fixed number of decimal points and precicely store them with a restricted range.

    \end{answer}

  \item Identify a user-defined ordinal type in the Java programming
    language.

  \begin{answer}

    Java has enumeration types so long as it is after Java 1.4.

    \end{answer}

  \item Mathematicians and programmers might have different ideas
    about the precedence of Boolean operators. Explain.

  \begin{answer}

   Mathematicians have the OR and AND opperators with equal precedence while programmers have AND with a higher precedence than OR.

    \end{answer}

  \item Programmers should use \verb+===+ rather than \verb+==+ to
    test the equality of the values of two expressions in JavaScript. Why?

  \begin{answer}

    The  \verb+===+ over \verb+==+ prevents coersion of one type into another such as a string that is numbers to be coerced into a number rather than being a string containing the number.

    \end{answer}

  \item Describe a hazard of allowing short-circuited evaluation
    of expressions and side effects in expressions at the same time.

  \begin{answer}

    Short-circuit evaluations used in an expression with part of the expression being a side effectthat is not evaluated. The side effect will only occur in complete evaluation of the whole ezxpression. This causes the side effects in expressions to  allow for subtle errors.

    \end{answer}

  \item Briefly describe the three steps in the mark-sweep algorithm
    for garbage collection.

  \begin{answer}

   The first are cells in the heap have indicators set to garbage. The second is when every pointer in the program is traced to the heap and reachables are marked to not be garbage. The third is that all cells not marked as being used are returned to being useable space.

    \end{answer}

  \item What led Yukihiro Matsumoto to create the Ruby programming language?

  \begin{answer}

    There was a dissatisfaction with Perl and Python and the designers as both supported object oriented programing but neither were purely object oriented programing.

    \end{answer}

  \item What did Microsoft aim to achieve with its development of the
    C\# language?

  \begin{answer}

    C\# was aimed to be a component-based software development in the .NET framework.

    \end{answer}

  \end{enumerate}


