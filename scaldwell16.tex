
\chapter{Sam Caldwell}

\begin{enumerate}
  \item Most programming languages require the use of brackets to
    enclose the index in a reference to an element of an array.
  \begin{enumerate}
    \item Identify a language the requires the use of parentheses
      to enclose the index in a reference to an element of an array.
    \item Why did the designers of the language choose parentheses
      rather than brackets?
    \end{enumerate}

  \begin{answer}

  \begin{enumerate}
    \item Ada
    \item The designers of Ada specifically chose parentheses to
      enclose subscripts so there would be uniformity between array
      references and function calls in expressions, in spite of
      potential readability problems
    \end{enumerate}

    \end{answer}
    
  \item What is the relationship between a lexeme and a token?

  \begin{answer}

    A token represents functional groups of lexemes, which are
    representations of the lowest level of syntactic units

    \end{answer}

  \item
  \begin{enumerate}
    \item What kind of symbols are found at the internal nodes of a
      parse tree?
    \item What kind of symbols are found at the leaves of a parse tree?
    \end{enumerate}

  \begin{answer}

  \begin{enumerate}
    \item internal nodes have non-terminal symbols
    \item leaf nodes have terminal symbols
    \end{enumerate}

    \end{answer}


  \item One of the most significant contributions from the developers
    of ALGOL 60 also limited the success of that language. What was
    that contribution?

  \begin{answer}

    "Ironically, one of the most important contributions to computer
    science associated with ALGOL 60, BNF, was also a factor in its
    lack of acceptance. Although BNF is now considered a simple and
    elegant means of syntax descrip- tion, in 1960 it seemed strange
    and complicated."  pg. 57 (10th edition)

    \end{answer}

  \item What problem were the creators of Common LISP trying to solve?

  \begin{answer}

    The developers of Common LISP were trying to solve issues with
    portability among programs written in various dialects.

    \end{answer}

  \item What is an ambiguous context free grammar?

  \begin{answer}

    "A grammar that generates a sentential form for which there are
    two or more distinct parse trees is said to be ambiguous"

    \end{answer}

  \item Contrast the complexity of algorithms that can parse strings
    that conform to the most general kinds of context free grammars
    and the complexity of the algorithms that can parse strings that
    conform to the grammars of programming languages?

  \begin{answer}

    Parsing algorithms for unambiguous grammars are complicated and
    inneficient. In fact, " the complexity of such algorithms is
    $O(n^3)$"(pg. 180). On the other hand, the algorithms used for
    context free grammars are closer to the level of $O(n)$, which means
    the time they take is linearly related to the length of the string
    to be parsed. This is vastly more efficient than $O(n^3)$
    algorithms.

    \end{answer}

  \item Java represents characters with Unicode. It is the first
    widely used programming language with this feature. What is the
    significance of this feature?

  \begin{answer}

    The previously used ASCII was becoming obsolete with the
    globalization of business and the need for computers to
    communicate around the world. Java quickly becomes a global coding
    language based on it's acceptance of unicode - which includes
    (among other things) the Cryllic alphabet.

    \end{answer}

  \item How does the binary coded decimal type differ from the
    floating point type?

  \begin{answer}

    "Decimal types have the advantage of being able to precisely store
    dec- imal values, at least those within a restricted range, which
    cannot be done with floating-point"

    \end{answer}

  \item Identify a user-defined ordinal type in the Java programming
    language.

  \begin{answer}

    "There are two user-defined ordinal types that have been supported
    by programming languages: enumeration and subrange."

    \end{answer}

  \item Mathematicians and programmers might have different ideas
    about the precedence of Boolean operators. Explain.

  \begin{answer}

    Because in programming, arithmetic expressions can be the operands
    of relational expressions, and relational expressions can be the
    operands of Boolean expressions. For example the '=' sign - in
    mathmatics is signifies that two sides of an equation are the same
    value, wheras in programming it signifies the changing of value of
    a variable (or other things).

    \end{answer}

  \item Programmers should use \verb+===+ rather than \verb+==+ to
    test the equality of the values of two expressions in JavaScript. Why?

  \begin{answer}

    Because the '==' operator uses coercions to achieve equality,
    '===' is testing for it.

    \end{answer}

  \item Describe a hazard of allowing short-circuited evaluation
    of expressions and side effects in expressions at the same time.

  \begin{answer}

    "Suppose that short-circuit evaluation is used on an expression
    and part of the expres- sion that contains a side effect is not
    evaluated; then the side effect will occur only in complete
    evaluations of the whole expression."

    \end{answer}

  \item Briefly describe the three steps in the mark-sweep algorithm
    for garbage collection.

  \begin{answer}

    First, all cells in the heap have their indicators set to indicate
    they are garbage.  Second, Every pointer in the program is traced
    into the heap, and all reachable cells are marked as not being
    garbage Third, All cells in the heap that have not been
    specifically marked as still being used are returned to the list
    of available space
    \end{answer}

  \item What led Yukihiro Matsumoto to create the Ruby programming language?

  \begin{answer}

    "The motivation for Ruby was dissatisfaction of its designer with
    Perl and Python. Although both Perl and Python support
    object-oriented programming,14 nei- ther is a pure object-oriented
    language, at least in the sense that each has primi- tive
    (nonobject) types and each supports functions."(pg. 100)

    \end{answer}

  \item What did Microsoft aim to achieve with its development of the
    \verb+C#+ language?

  \begin{answer}

   "The purpose of \verb+C#+ is to provide a language for component-based
    software development, specifically for such development in the
    .NET Framework. In this environment, components from a variety of
    languages can be easily com- bined to form systems"

    \end{answer}

  \end{enumerate}


