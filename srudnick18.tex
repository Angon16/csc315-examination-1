
\chapter{Spencer Rudnick}

\begin{enumerate}
  \item Most programming languages require the use of brackets to
    enclose the index in a reference to an element of an array.
  \begin{enumerate}
    \item Identify a language the requires the use of parentheses
      to enclose the index in a reference to an element of an array.
    \item Why did the designers of the language choose parentheses
      rather than brackets?
    \end{enumerate}

  \begin{answer}

  \begin{enumerate}
    \item Ada is an example of a programming language that uses
      parentheses to enclose indices to reference elements of arrays.
    \item The designers of Ada chose to use parentheses for enclosing
      indices, in spite of the fact that it makes code more difficult
      to read, because both array references and function calls map to
      an address in memory. An understanding of how a program runs on
      the hardware of computer reveals that this seemingly strange
      choice actually makes a lot of sense.
    \end{enumerate}

    \end{answer}
    
  \item What is the relationship between a lexeme and a token?

  \begin{answer}

    A token describes a category of lexemes. For example, all variable
    names, \textit{i, index, currentNode, etc.}, are \textit{lexemes}
    within the token \textit{Identifiers}. \"=\" has its own token,
    \textit{equal\_sign}. And so on, all down the line of reserved
    words, characters, and user-defined variables.

    \end{answer}

  \item
  \begin{enumerate}
    \item What kind of symbols are found at the internal nodes of a
      parse tree?
    \item What kind of symbols are found at the leaves of a parse tree?
    \end{enumerate}

  \begin{answer}

  \begin{enumerate}
    \item Nonterminal symbols are found at the internal nodes of a
      parse tree. Nonterminals include \textit{if()} statements and
      other function calls.
    \item Terminal symbols are found at the leaves of a parse
      tree. Terminals include lexemes and tokens, such as integer
      literals and variable names.
    \end{enumerate}

    \end{answer}


  \item One of the most significant contributions from the developers
    of ALGOL 60 also limited the success of that language. What was
    that contribution?

  \begin{answer}

    ALGOL 60 was the first language to be defined using what is now
    known as Backus-Naur Form, or BNF for short. BNF was a very
    significant contribution to the field of computer
    science. However, it did not catch on very quickly at the time of
    ALGOL 60's release, so few people adopted the language.

    \end{answer}

  \item What problem were the creators of Common LISP trying to solve?

  \begin{answer}

    The creators of Common LISP were trying to solve the problem of
    fragmentation in the LISP family, which made code portability very
    difficult. Common LISP combined elements of many other versions of
    LISP, creating a language which was portable, yet very complex.

    \end{answer}

  \item What is an ambiguous context free grammar?

  \begin{answer}

    A grammar for which sentences can have more than one valid parse tree.

    \end{answer}

  \item Contrast the complexity of algorithms that can parse strings
    that conform to the most general kinds of context free grammars
    and the complexity of the algorithms that can parse strings that
    conform to the grammars of programming languages?

  \begin{answer}

    Algorithms which parse strings conforming to general kinds of
    context free grammars function on $O(n)$.

    Algorithms which parse strings conforming to the grammars of a
    modern programming language usually function on the order of
    $O(n^3)$.

    \end{answer}

  \item Java represents characters with Unicode. It is the first
    widely used programming language with this feature. What is the
    significance of this feature?

  \begin{answer}

    Unicode represents characters from almost every natural language
    and number system. This allows information encoded in Unicode to
    be read on almost any machine (especially those running Java).

    \end{answer}

  \item How does the binary coded decimal type differ from the
    floating point type?

  \begin{answer}

    Decimals are precise definitions of a value, though they are
    usually only allowed to be in a range specified by the language
    and hardware. They contain a fixed decimal point with exact
    numbers defined on both sides.

    Floating points are approximations of a value. They are not
    precise, but they are able to represent a vastly greater span of
    values. Floating points are made up of a sign bit, an exponent
    byte (or two), and between three and eight bytes defining a
    fraction.

    \end{answer}

  \item Identify a user-defined ordinal type in the Java programming
    language.

  \begin{answer}

    \textit{integer} is a user-defined ordinal type in Java.

    \end{answer}

  \item Mathematicians and programmers might have different ideas
    about the precedence of Boolean operators. Explain.

  \begin{answer}

    In boolean algebra, the AND and OR operators have the same
    precedence. However, C-based languages assign a higher precedence
    to AND than OR, likely because of an erroneous associations
    between arithmetic multiplication with the AND function, and
    arithmetic addition with the OR function.

    \end{answer}

  \item Programmers should use \verb+===+ rather than \verb+==+ to
    test the equality of the values of two expressions in JavaScript. Why?

  \begin{answer}

    In JavaScript, \verb+"=="+ allows the operands to be coerced (one
    is converted into the same type as the other for easy
    comparison). \verb+"7" == 7+ would return true. \verb+"==="+
    prevents coercion by the interpreter, so \verb+"7" === 7+ would
    return false.

    \end{answer}

  \item Describe a hazard of allowing short-circuited evaluation
    of expressions and side effects in expressions at the same time.

  \begin{answer}

    There is a possibility that the whole expression may not be
    parsed, and so the side effect which was intended to happen is
    skipped. This could result in small errors in which an integer
    which was supposed to be incremented is not! (gasp!)

    \end{answer}

  \item Briefly describe the three steps in the mark-sweep algorithm
    for garbage collection.

  \begin{answer}

    Step 1. All cells in the heap are marked as garbage.

    Step 2. Each cell is checked to see if it is reachable. If it is,
    it is marked as still in use (not garbage).

    Step 3. All cells still marked as garbage are destroyed (removed
    from the heap).

    \end{answer}

  \item What led Yukihiro Matsumoto to create the Ruby programming language?

  \begin{answer}

    Yukihiro Matsumoto created Ruby out of dissatisfaction with Perl
    and Python, specifically that they were not pure object-oriented
    languages.

    \end{answer}

  \item What did Microsoft aim to achieve with its development of the
    C\# language?

  \begin{answer}

    The purpose of C\# was to create a language which can combine
    components in any other language within the .NET framework (C\#,
    Visual Basic, .Net, Managed C++, F\#, and JScript .Net). Seems
    like they were trying to incorporate some of the Unix philosophy
    we learned about on the first day!

    \end{answer}

  \end{enumerate}


